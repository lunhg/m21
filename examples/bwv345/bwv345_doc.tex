\subsection{BWV345}

Com os comandos \verb|python main.py -i bwv345 -f 6| realiza a seguinte saída:

\begin{quote}
{%
\parindent 0pt
\noindent
\ifx\preLilyPondExample \undefined
\else
  \expandafter\preLilyPondExample
\fi
\def\lilypondbook{}%
\includegraphics{./bwv345/bd/lily-970a89f3-1}%
\ifx\betweenLilyPondSystem \undefined
  \linebreak
\else
  \expandafter\betweenLilyPondSystem{1}%
\fi
\includegraphics{./bwv345/bd/lily-970a89f3-2}%
\ifx\betweenLilyPondSystem \undefined
  \linebreak
\else
  \expandafter\betweenLilyPondSystem{2}%
\fi
\includegraphics{./bwv345/bd/lily-970a89f3-3}%
\ifx\betweenLilyPondSystem \undefined
  \linebreak
\else
  \expandafter\betweenLilyPondSystem{3}%
\fi
\includegraphics{./bwv345/bd/lily-970a89f3-4}%
% eof

\ifx\postLilyPondExample \undefined
\else
  \expandafter\postLilyPondExample
\fi
}
\end{quote}

Percebendo que a tendência dos materiais possuirem uma direcionalidade para a tessitura dos graves, apliquei uma regra improvisada de contratempos e retardos harmônicos, que pudesse refletir esse fluxo harmônico para os graves, sem uma resolução bem definida, mas com a sensação de centro tonal; escolhi a nota ré como ponto de referência para novos ataques, como se um impulso \footnote{Impulso é um termo utilizado em DSP para descrever um ruído de banda limitada emitido em um curto espaço de tempo; o som resultadnte na sala onde foi emitido é chamado de \emph{resposta de impulso}, com espectros singulares para cada ambiente.} reverberasse em um espaço acústico. Ao final é interessante notar a finalização por $V~\Leftrightarrow~I$, o que musicalmente me pareceu uma estrutura cadencial.
 

\begin{quote}
{%
\parindent 0pt
\noindent
\ifx\preLilyPondExample \undefined
\else
  \expandafter\preLilyPondExample
\fi
\def\lilypondbook{}%
\includegraphics{./bwv345/97/lily-0b97f347-1}%
\ifx\betweenLilyPondSystem \undefined
  \linebreak
\else
  \expandafter\betweenLilyPondSystem{1}%
\fi
\includegraphics{./bwv345/97/lily-0b97f347-2}%
\ifx\betweenLilyPondSystem \undefined
  \linebreak
\else
  \expandafter\betweenLilyPondSystem{2}%
\fi
\includegraphics{./bwv345/97/lily-0b97f347-3}%
% eof

\ifx\postLilyPondExample \undefined
\else
  \expandafter\postLilyPondExample
\fi
}
\end{quote}