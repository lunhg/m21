\documentclass[12pt]{article}

\usepackage{sbc2003}

\usepackage{graphicx,url}

\usepackage[brazil]{babel}   
\usepackage[utf8]{inputenc}

% Para estilização dos códigos
\usepackage{minted}

% Para lista em linhas
\usepackage{paralist}

% Identar primeiros parágrafos
\usepackage{indentfirst}

%TODO
\usepackage[colorinlistoftodos]{todonotes}

\usepackage{minted}

\usepackage{array}

%---
% Pacote para listas em uma linha
%---
\usepackage{paralist}

\title{Colagem, recorte e erros em um processo composicional utilizando o Music21.}

\author{Guilherme Lunhani\inst{1}}

\address{Instituto de Artes e Design --
         Universidade Federal de Juiz de Fora \\
         Juiz de Fora, MG
         \email{gcravista@gmail.com}
}

\begin{document}

\maketitle

\begin{abstract}
This article describes a case study in Computer Generated Assistance, for music analysis and didactic composition. A Python command was programmed to automate routines based on \cite{music21_2015} library, such: selection, cluster and fragmentation from a document in J.S.Bach's corpus. Some compositional exercises were used to test a operation, called \emph{glitch}, in this corpus. In addition, \cite{musescore_2015} and \cite{lilypond_2015} were used to edit and diagram scores. At the end, comment on bugs, compositional problems and future plans of composition.
\end{abstract}

\begin{resumo}
Este artigo descreve um estudo de caso em Assitência Gerada por Computador para análise musical e composição didática. Um comando {Python} foi programado para automatizar rotinas da biblioteca \cite{music21_2015} como: seleção, agrupamento e fragmentação de um documento no corpus bachiano. Alguns exercícios composicionais foram usados para testar uma operação \emph{glitch}, neste corpus. Adicionalmente,  \cite{musescore_2015} e \cite{lilypond_2015} foram utilizados para edição. Ao final comentarei \emph{bugs}, problemas composicionais  e planos futuros de composição.
\end{resumo}

\section{Introdução}

Este artigo trata de um protótipo, \emph{m21.py},  desenvolvido a partir da biblioteca \cite{music21_2015}, para composição e análise musical. Na seção \ref{sec:trabalhos} contextualizo o que me levou a elaborar o programa.  Na seção \ref{sec:metodo} descrevo as tarefas realizadas para desenvolvimento do \emph{m21}. Na seção \ref{sec:m21} a utilização do \emph{m21}. 

Este código capacitou a produção de um número considerável de exercícios composicionais para piano. A intenção é oferecer uma ferramenta quase imediatista de material pré-composicional para exercícios criativos. Pode ser útil em cursos de Composição Assistida por Computador, em universidades ou em oficinas de arte. Por economia de espaço,apresentarei na seção \ref{sec:resultados} um exemplo. Na seção \ref{sec:conclusao} discuto problemas técnicos do \emph{software}, problemas composicionais. Na seção 7 planos futuros.
\section{Trabalhos relacionados}\label{sec:trabalhos}

Durante uma pesquisa de mestrado, sobre \emph{live coding}, tive contato com o Music21 que, segundo \cite{soares_luteria_2015}:

\begin{quote}
É uma biblioteca projetada para trabalhar com manipulação e análise de corpus de arquivos partituráveis. Prepara a conversão entre diversos arquivos de dados musicais. (\ldots) Music21 tem uma abordagem voltada para uma "musicologia assistida por computador" e já tem incorporada em suas classes algumas ferramentas comuns a esta prática como: numeração de grau funcional de acorde, numeração de classes de altura usando a classificação de Allen Forte : a implementação dos algoritmos de detecção de tonalidade elaborado por Krumhansl (1990) e aperfeiçoada por Temperley (2001), busca de padrões como transposições e inversões e outros.\cite[p.~71-72]{soares_luteria_2015}
\end{quote}

Ao invés de compor adicionando informações aos dados musicais, busquei na \emph{Estética do Erro} \cite{cascone_aesthetics_2000} os procedimentos básicos para composição, explicados com mais detalhes na seção \ref{sec:metodo}. Resultados sonoros procedentes da colagem e do erro dependem muito do \emph{input}. Aplicar o mesmo algoritmo de erro para diferentes materiais, ou, aplicar diferentes erros para um único material, não resulta em um produto homogêneo.

Busquei então utilizar diferentes documentos do \emph{corpus}, e um mesmo erro para realizar um tipo de música que, segundo \cite[p.~18]{soares_luteria_2015}, existia intensamente ``antes da preocupação imediata com os timbres ou da era das manipulações de amostras sonoras - e de certa maneira ainda proto-serialista. Uma música por vezes chamada politonal, polimodal ou usando o termo de Straus (2004): pós-tonal.''. Sendo um tipo de composição que não é nova, mas relevante do ponto de vista histórico e didático-composicional,  busquei elaborar uma ferramenta para ser usada em processos criativos musicais. 

\subsection*{Questões pessoais}

Não deixo de mencionar uma antiga conversa com o compositor Franscisco Zmekhol Nascimento de Oliveira, que levou-me a compor segundo regras arbitrárias, na contingência do momento. Isto é, uma música para cada dia da existência. 

Ao realizar o mesmo procedimento de composição de um mesmo documento do \emph{corpus} bachiano, o material pré-composicional resultante deve ser diferente de qualquer outro. 

Para realizar tecnicamente, valores não-determinados em operações determinísticas são utilizados.

Para compor-improvisar com os materiais resultantes, apliquei princípios de articulação do som pelo silêncio, elaborados por rascunhos de partituras-planimétricas, utilizadas por \cite{koellreutter_introducao_1987}.
\section{Metodologia}\label{sec:metodo}

\subsection*{Organização dos códigos}

O programa foi separado em três arquivos: \begin{inparaenum}[\itshape i)\upshape]
\item um binário em \emph{Python} que realiza tarefas gerais da linha de comando (\emph{m21});
\item rotinas do Music21 (\emph{m21utils.py}); e 
\item um para rotinas externas (\emph{tools.py})
\end{inparaenum}\footnote{Todos códigos, exemplos e documentação estão disponíveis \url{https://www.github.com/jahpd/m21}.}.

\subsection*{Categorização do software}

Nas palavras de \cite[p.~x-xiii]{cope_prefacio_2008}, o \emph{m21} pode ser classificado como uma ferramenta para uma Assistência Gerada por Computador (\emph{Computer Generated Assistance} ou CGA). Dentro das sub-categorias de CGA propostas por Cope, o \emph{m21} pode ser incuído nos três modos abaixo:\begin{inparaenum}[\itshape 1)\upshape]
\item uso de uma Linguagem de Programação em texto (PLs)(\emph{Programming Languages}) ao invés de uma linguagem de programação visual (VPL); 
\item o material partitural é gerado para performance humana ao invés de uma performance eletroacústica;
\item abordagens baseadas em regras (\emph{Rules Based}) e Dirigido a dado (\emph{Data-Driven}) são usados para modificar um material existente.
\end{inparaenum}

\subsection*{Método de composição}

Em geral, o procedimento de composição se deu a partir da execução de um comando \emph{m21} com argumentos explicados na seção \ref{sec:resultados}: \begin{inparaenum}[\itshape i)\upshape]
\item gerar um material musical com base em uma colagem de uma obra no corpus do Music21;
\item o material colado será submetido a subtração de compassos, aleatoriamente;
\item dos compassos restantes, quaisquer notas serão agrupadas como um evento;
\item destes agrupamentos, as oitavas serão embaralhadas; e
\item do bloco harmônico resultante, pode ser que alguma nota fique deslocada, gerando figuras.
\end{inparaenum}

Busquei notar densidades, fraseado e pontos de finalização ``naturais'' do material harmônico resultante Tais parâmetros eram editados no MuseScore, sendo que algumas interferências não previstas foram incluídas. Após edição, ocorreu o processo de editoração no Lilypound para melhor visualização dos resultados.

Por último, o material pré-composicional foi editado no \cite{musescore_2015} e posteriorente diagramado no \cite{lilypond_2015}.
\input{./04-M21}
\section{Resultados}\label{sec:resultados}

Comandos foram utilizados para a estruturação de ``Corais'', pequenas peças didáticas de, por colagem de materiais da bachianos, recorte por técnica \emph{glitch}\footnote{Aplicação de recortes indeterminados  sobre um material pré existente.}, e readequações de discurso harmônico pós-tonal. Resultados foram possíveis com a implementação da a opção \verb|--CAC| ou \verb|-C|, como no código abaixo. No entanto esta é apenas uma \emph{flag} indicativa que um material será reconfigurado. Inclui, até o momento, a \emph{flag} \verb|--glitch| ou \verb|-g|, que recorta o material fornecido.

\begin{listing}
\begin{minted}[linenos,frame=leftline,fontsize=\footnotesize]{python}
./main.py --show --CAC --composer bach --index bwv1 --glitch 2
./main.py -S -C -c bach -i bwv1 -g 2
\end{minted}
\end{listing}

\subsection{BWV1}

O sexto movimento de \emph{Wie schün leucht der Morgenstern} (Cantata para a festa da Anunciação, 1725, ver figura \ref{fig:bwv_frag})\footnote{Vídeo disponível em \url{https://www.youtube.com/watch?v=POe2fBjbswA}.} foi utilizado segundo procedimentos explicados na seção \ref{sec:metodo}. Foi gerado um conjunto de simultanóides\footnote{Utilizo aqui a nomenclatura de \cite{koellreutter_introducao_1987} para identificar blocos harmônicos.}, apresentado na figura \ref{fig:bwv1_extract}.

\begin{figure}[h]
  \centering
  {%
\parindent 0pt
\noindent
\ifx\preLilyPondExample \undefined
\else
  \expandafter\preLilyPondExample
\fi
\def\lilypondbook{}%
\input{../examples/bwv1/c7/lily-6fc7e149-systems.tex}
\ifx\postLilyPondExample \undefined
\else
  \expandafter\postLilyPondExample
\fi
}

  \caption{Sequência de simultanóides gerados. \textbf{Fonte}: Autor.}
  \label{fig:bwv1_extract}
\end{figure}

 Cada bloco harmônico são notas de um determinado compasso, escolhido ao acaso pelo programa, comprimidas em um único evento. Algumas notas podem ter sido omitidas por erros de codificação no \emph{script} Python. Mas isso é coerente com o princípio estético. É também  interessante observar uma direcionalidade da tessitura, que vai da região média aos graves, percebida após repetidos usos do comando descrito. 

Realizei algumas intervenções com este material: \begin{itemize}
\item manutenção da ordem dos blocos;
\item escolha de pontos que delimitam fraseados (silêncio como articulado de frases;);
\item observação de densidades na quantidade de notas como fator para o fim do fraseado;
\item dinâmicas como um dispositivo de ênfase do fraseado harmônico; 
\item a modificação de uma oitava de dois si bemoís para um intervalo de sétima ou nona de si bequadro e si bemol compasso, inserindo um intervalo de sétima com outro sib;  
\item deslocamento ou subtração de elementos da figura \ref{fig:bwv1_extract}.
\end{itemize}

A peça finalizada está na figura \ref{fig:bwv1}.

 \begin{figure}
  \centering
  {%
\parindent 0pt
\noindent
\ifx\preLilyPondExample \undefined
\else
  \expandafter\preLilyPondExample
\fi
\def\lilypondbook{}%
\input{../examples/bwv1/d8/lily-ef199858-systems.tex}
\ifx\postLilyPondExample \undefined
\else
  \expandafter\postLilyPondExample
\fi
}

  \caption{Peça resultante das intervenções. \textbf{Fonte}: autor.}
  \label{fig:bwv1}
\end{figure}

As ferramentas analíticas auxiliaram a observação de semelhanças e diferenças entre uma peça original  e a peça variada. Por exemplo, com BWV1 de J.S.Bach, o histograma da figura \ref{fig:pitch-class-bwv1-histogram} revela fatos analíticos comuns, como uma ênfase do \emph{pitch class} 5, Fá, I grau; e depois \emph{pitch class} 0, Dó, V grau, seguido de outras classes, como Lá (III grau) e Sol (ii grau ou V/V); outros, como Ré (vi grau) e Si bemol, possuem a mesma quantidade. Por último Mi (vii grau) e si natural possuem menor número, talvez como dispositivos cadenciais(como por exemplo na ``modulação'' Fá$\Rightarrow$Dó do quinto compasso da figura \ref{fig:bwv_frag}).

A partir da artesania do material pré-composicional gerado por fragmentação do BWV1, notei, no histograma da figura \ref{fig:pitch-class-bwv1-histogram-2} que algumas proporções foram mantidas de maneira aproximada. A quantidade de eventos diminui drasticamente. Mesmo com a segmentação da peça, um centro tonal ainda pode ser notado (embora de maneira bastante ambígua).

\begin{figure}[!h]
  \centering
  \includegraphics[scale=0.71]{../analysis/bwv1/pitch-class-1.png}
  \caption{Histograma de \emph{pitch-class} da peça feita, utilizando o comando \emph{./main.py -x examples/bwv1/bwv1.xml --plot-histogram-pitch-class}}
    \label{fig:pitch-class-bwv1-histogram-2}
\end{figure}



\section{Conclusão}\label{sec:conclusao}

Duas funcionalidades do Music21 foram observadas: Musicologia Assistida por Computador  e Composição Assistida por Computador (CAC).  Da documentação e do corpus do \emph{software}, programei rotinas que lidassem com aquilas tarefas que considerei demasiadamente laboriosas. Para isso proponho comandos que  podem auxiliar em tarefas cotidianas, composicionais ou analíticas. 

Da rotina para composição/análise, cito a busca no corpus, de uma obra específica ou obras pelo nome do compositor. Das rotinas analíticas, enumero \begin{inparaenum}[\itshape a)\upshape]
\item identificação dos graus
\item estruturação intervalar de blocos harmônicos
\item plotamento de histogramas de classes de altura. Das rotinas composicionais, enumerei procedimentos para uma composição por fragmentação (\emph{glitch}):
\item colagem de fragmentos de uma partitura
\item compressão de melodias em blocos harmônicos dos fragmentos resultantes
\item troca de oitavas com as notas deste bloco
\item possível fragmentação do bloco resultante em figuras ou arpejos
\end{inparaenum}Existe adicionalmente um $bug$ que ocorre por fatores lógicos, próprio da atividade de programação realizada, mas ainda não resolvido. Possivelmente o compositor pode lidar com dados nulos. Caso a peça de entrada seja muito grande, pode acontecer do programa não responder. 

Foi observado uma versatilidade de materiais pré-composicionais gerados, bem como a possibilidade de improvisar com estes materiais. O exemplo apresentado na seção \ref{sec:resultados} foi feito em algumas horas. Com isso espero oferecer aos docentes e discentes em composição musical uma ferramenta alternativa para atividades diáticas em processos criativos.

\section{Planos Futuros}

Correção do \emph{bug} e implementação de um novo comando que fragmente várias obras do corpus em um único material pré-composicional (comando \verb|--poop| ou \verb|-p|). Continuação de novas composições para o ciclo.

\section{Agradecimentos}

Ao Guilherme Rafael Soares por apresentar a biblioteca \emph{music 21}. Aos desenvolvedores do \emph{musescore} e \emph{lilypound}. FAPEMIG pelo financiamento da pesquisa.


\clearpage\clearpage
\bibliographystyle{sbc}
\bibliography{article}

\end{document}