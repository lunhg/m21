\section{Trabalhos relacionados}\label{sec:trabalhos}

Este trabalho iniciou com a investigação do Music21 que, segundo \cite{soares_luteria_2015}:
\ \\
\begin{quote}
É uma biblioteca projetada para trabalhar com manipulação e análise de corpus de arquivos partituráveis. Prepara a conversão entre diversos arquivos de dados musicais.(\ldots) Music21 tem uma abordagem voltada para uma "musicologia assistida por computador" e já tem incorporada em suas classes algumas ferramentas comuns a esta prática como: numeração de grau funcional de acorde, numeração de classes de altura usando a classificação de Allen Forte : a implementação dos algoritmos de detecção de tonalidade elaborado por Krumhansl (1990) e aperfeiçoada por Temperley (2001), busca de padrões como transposições e inversões e outros.\cite[p.~71-72]{soares_luteria_2015}
\end{quote}
\ \\

Por um lado, o desenvolvimento do \emph{m21} busca encorarjar a utilização de métodos numéricos aplicados à Análise Musical\footnote{Por exemplo, a plotagem de um gráfico contendo histogramas de classes de altura de uma peça.}. Por outro, encoraja a transposição de técnicas, nas palavras de  \cite{cascone_aesthetics_2000}, pós-digitais. Isto é, uma tradução de técnicas do universo eletroacústico para o universo da notação em partitura. No caso do universo eletroacústico, Cascone esclarece:
\ \\

\begin{quote}
A estética pós-digital foi desenvolvida em parte como resultado de uma experiência imersiva, de um trabalho em ambientes repletos de tecnologia digital: computadores, zumbido de \emph{fans}\footnote{``Ventiladores'' do \emph{cooler}.}, impressoras a laser produzindo documentos, sonificação de interfaces de usuário, e o barulho abafado de discos rígidos. Mas mais especificamente, surgia das falhas da tecnologia digital que surgiram neste novo trabalho: falhas, \emph{bugs}, aplicação de erros, quebras de sistema, clipagem, serrilhamento, distorção, quantização, e mesmo o ruído de fundo de placas de som são materiais crus que compositores procuram incorporar em sua música.\cite[p.~393]{cascone_aesthetics_2000}\footnote{Tradução de \emph{The "post-digital" aesthetic was developed in part as a result of the immersive experience of working in environments suffused with digital technology: computer fans whirring, laser printers churning out documents, the sonification of user-interfaces, and the muffled noise of hard drives. But more specifically. it is from the "failure" of digital technology that this new work has emerged: glitches, bugs, application errors, system crashes, clipping, aliasing, distortion, quantization noise, and even the noise floor of computer sound cards are the raw materials composers seek to incorporate into their music.}}
\end{quote}
\ \\

A tradução do presente artigo, pode ser relatada como uma experiência de \emph{aplicação de erros}: se cruzarmos os conceitos do processamento de áudio digital, com os conceitos da notação tradicional em partituras, podemos simular erros em peças do cânone musical. Estas últimas podem ser encontradas como arquivos de computador, em formato \cite{musicxml_2015} no \cite{music21_2015}. Ao selecionarmos uma, e aplicarmos erros como recorte de compassos, agrupamento de notas, desmembramento de acordes, é possível obter harmonias características daquele período que, segundo Soares, existia:

\ \\
\begin{quote}
(\ldots) antes da preocupação imediata com os timbres ou da era das manipulações de amostras sonoras - e de certa maneira ainda proto-serialista. Uma música por vezes chamada politonal, polimodal ou usando o termo de Straus (2004): pós-tonal.''.\cite[p.~18]{soares_luteria_2015}
\end{quote}
\ \\

Não sendo um tipo composição nova, é relevante do ponto de vista didático. Neste sentido foi programada uma ferramenta para ser usada em processos criativos musicais, considerando fatores pedagógicos.

\subsection*{Questões diversas}

O compositor Franscisco Zmekhol Nascimento de Oliveira sugeria anotar em partituras, peças com um mesmo processo, mas que segundo a contingência do momento, são diferenciadas por seus eventos musicais. Esta abordagem encontrou respaldo nas estratégias composicionais propostas por \cite{koellreutter_introducao_1987}. Após a geração dos materiais no \emph{m21}, uma improvisação de partituras-planimétricas auxiliou articular sons e silêncios, possibilitando fraseados  semelhantes aos corais de J.S. Bach.


