\begin{table}
\caption{Tabela das opções utilizadas para produção e análise de um arquivo partiturável. \textbf{Fonte}: autor.}
\small
    \begin{tabular}{| p{3cm} | p{3cm} | p{2.25cm} | p{5.75cm} |}
    \hline 
    \hline 

    \textbf{Nome} & 
    \textbf{Comando} &
    \textbf{Abreviação} &
    \textbf{Execução}\\

    \hline

    \textbf{Compositor} 
    & \verb|--composer|
    & \verb|-c|             
    & Campo de procura no corpus pelo nome de um compositor. Usado em conjunto com a opção ``Index''. \\
    \hline

    \textbf{Index} 
    & \verb|--index|    
    & \verb|-i|             
    & Indexação de uma peça (catálogo, p.e., bwv123). Usado em conjunto com a opção ``Compositor'' \\
    \hline

    \textbf{Composição Assistida por Computador} 
    & \verb|--CAC|  
    & \verb|-C|             
    & \emph{Flag} indicativa de uma operação de transformação em uma partitura. Usada em conjunto, as opções ``Compositor'' e ``Index''.\\
    \hline

     \textbf{Glitch} 
    & \verb|--glitch|      
    & \verb|-g|              
    & \emph{Flag} indicativa do tipo de operação de transformação. A peça é desorganizada e verticalizada em blocos harmônicos, de maneira randômica, limitada apenas por regras de tessitura do piano. \\
    \hline

    \textbf{Apresentar em um editor de partituras} 
    & \verb|--Show|       
    & \verb|-S|             
    & \emph{Flag} indicativa que, o resultado obtido será executado em um editor de partituras apropriado, no caso deste trabalho, o \cite{musescore_2015}.\\
    \hline
    
    \textbf{Plotar gráficos analíticos} 
    & \verb|--plot-*|       
    & -                                          
    & \emph{Flag} indicativa que um gráfico analítico será gerado. O Símbolo ``*'' representa a diversidade dos tipos de gráficos possíveis. \\
    \hline
    \hline
   
    \end{tabular}
\label{tab:comandos}
\end{table}