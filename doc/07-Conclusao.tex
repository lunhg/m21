\section{Conclusão}\label{sec:conclusao}

Duas funcionalidades do Music21 foram observadas: Musicologia Assistida por Computador  e Composição Assistida por Computador (CAC).  Da documentação e do corpus do \emph{software}, programei rotinas que lidassem com aquilas tarefas que considerei demasiadamente laboriosas. Para isso proponho comandos que  podem auxiliar em tarefas cotidianas, composicionais ou analíticas. 

Da rotina para composição/análise, cito a busca no corpus, de uma obra específica ou obras pelo nome do compositor. Das rotinas analíticas, enumero \begin{inparaenum}[\itshape a)\upshape]
\item identificação dos graus
\item estruturação intervalar de blocos harmônicos
\item plotamento de histogramas de classes de altura. Das rotinas composicionais, enumerei procedimentos para uma composição por fragmentação (\emph{glitch}):
\item colagem de fragmentos de uma partitura
\item compressão de melodias em blocos harmônicos dos fragmentos resultantes
\item troca de oitavas com as notas deste bloco
\item possível fragmentação do bloco resultante em figuras ou arpejos
\end{inparaenum}Existe adicionalmente um $bug$ que ocorre por fatores lógicos, próprio da atividade de programação realizada, mas ainda não resolvido. Possivelmente o compositor pode lidar com dados nulos. Caso a peça de entrada seja muito grande, pode acontecer do programa não responder. 

Foi observado uma versatilidade de materiais pré-composicionais gerados, bem como a possibilidade de improvisar com estes materiais. O exemplo apresentado na seção \ref{sec:resultados} foi feito em algumas horas. Com isso espero oferecer aos docentes e discentes em composição musical uma ferramenta alternativa para atividades diáticas em processos criativos.
